\documentclass{article}

\usepackage[utf8]{inputenc}
\usepackage[russian]{babel}
\usepackage[a4paper, margin=1in]{geometry}
\usepackage{graphicx}
\usepackage{amsmath}
\usepackage{wrapfig}
\usepackage{multirow}
\usepackage{mathtools}
\usepackage{pgfplots}
\usepackage{pgfplotstable}
\usepackage{setspace}
\usepackage{changepage}
\usepackage{caption}
\usepackage{csquotes}
\usepackage{hyperref}
\usepackage{listings}

\pgfplotsset{compat=1.18}
\hypersetup{
  colorlinks = true,
  linkcolor  = blue,
  filecolor  = magenta,      
  urlcolor   = darkgray,
  pdftitle   = {
    math-tool-report-approx-smirnov-victor-p32131
  },
}

\definecolor{codegreen}{rgb}{0,0.6,0}
\definecolor{codegray}{rgb}{0.5,0.5,0.5}
\definecolor{codepurple}{rgb}{0.58,0,0.82}
\definecolor{backcolour}{rgb}{0.99,0.99,0.99}

\lstdefinestyle{codestyle}{
  backgroundcolor=\color{backcolour},   
  commentstyle=\color{codegreen},
  keywordstyle=\color{magenta},
  numberstyle=\tiny\color{codegray},
  stringstyle=\color{codepurple},
  basicstyle=\ttfamily\footnotesize,
  breakatwhitespace=false,         
  breaklines=true,                 
  captionpos=b,                    
  keepspaces=true,                 
  numbers=left,                    
  numbersep=5pt,                  
  showspaces=false,                
  showstringspaces=false,
  showtabs=false,                  
  tabsize=2
}

\lstset{style=codestyle}

\begin{document}

\begin{titlepage}
    \begin{center}
        \begin{spacing}{1.4}
            \large{Университет ИТМО} \\
            \large{Факультет программной инженерии и компьютерной техники} \\
        \end{spacing}
        \vfill
        \textbf{
            \huge{Теория Вероятности.} \\
            \huge{Практическая работа №6.} \\
        }
    \end{center}
    \vfill
    \begin{center}
        \begin{tabular}{r l}
            Группа:  & P32131                  \\
            Студент: & Смирнов Виктор Игоревич \\
            Вариант: & 16                      \\
        \end{tabular}
    \end{center}
    \vfill
    \begin{center}
        \begin{large}
            2023
        \end{large}
    \end{center}
\end{titlepage}

\section{Задача 1}

Дана выборка: $a = [0.27, 0.27, 0.28, 0.29, 0.3, 0.31, 0.33, 0.33, 0.35, 0.37]$.
$n = 10, \gamma = 0.99$

$A = mean = 0.31$

$S = \sqrt{D} = \sqrt{\frac{\sum_{i=1}^n(a_i - A)^2}{n - 1}}$

$t = student(n - 1, (1 + \gamma) / 2) = 3.25$, посмотрели в таблице.

$(A - \frac{tS}{\sqrt{n}}, m + \frac{tS}{\sqrt{n}}) = (0.275, 0.345)$

\section{Задача 2}

Дано: $
n = 64, 
A = mean(a) = 5452.8 / n = 85.2, 
S = \sqrt{\frac{973.44}{n}} = 3.9,
\gamma = 0.9
$

$(1 + gamma) / 2 = 0.95$

$(1.65 + 1.64) / 2 = 1.645$

$(A - \frac{tS}{\sqrt{n}}, A + \frac{tS}{\sqrt{n}}) = (84.398, 86.002)$

\section{Задача 3}

Дано: $c = [112, 168, 130, 69, 32, 5, 1, 1]$

$n = sum(c) = 518$

$A = mean(c) = 1.5463320463320462$

\begin{lstlisting}[
    language={Bash},
    caption={Таблица сырая}
]
0       112     0.21302791908795654     110.34846208756149
1       168     0.3294118980491374      170.63536318945316
2       130     0.254690087198223       131.9294651686795
3       69      0.1312784812392385      68.00225328192553
4       32      0.0507500306335087      26.288515868157507
5       5       0.01569527974418551     8.130154907488095
6       1       0.004045019007430049    2.0953198458487656
7       1       0.00089356321703019     0.4628657464216384
\end{lstlisting}

\begin{lstlisting}[
    language={Bash},
    caption={Таблица исправленная}
]
0       112     0.21302791908795654     110.34846208756149
1       168     0.3294118980491374      170.63536318945316
2       130     0.254690087198223       131.9294651686795
3       69      0.1312784812392385      68.00225328192553
4       32      0.0507500306335087      26.288515868157507
5       7       0.02063386196864575     10.688340499758498
\end{lstlisting}

Распределение пуассона считали по формуле

$p(i) = \frac{\lambda^i}{i!} e^{-\lambda}$

Получаем актуальное значение

$\chi^2 = \sum_{i=1}^n \frac{(b_i - b^*_i)^2}{b^*_i} = 2.621938350796128$

Теперь вычисляем ожидаемое значение

$k = n, l = 1, k - l - 1 = 4, 1 - \alpha = 0.99$

$\chi^2 = 13.3$

$2.622 < 13.3 \Rightarrow $ принимаем гипотезу.

\end{document}